%==== PACKAGES AND OTHER DOCUMENT CONFIGURATIONS  ====%
\documentclass{resume} % Use the custom resume.cls style
\usepackage[left=0.25in,top=0.25in,right=0.25in,bottom=0.25in]{geometry} % Document margins
\usepackage[T1]{fontenc}
\usepackage{xcolor}
\usepackage{lmodern}
\usepackage{fontawesome} % For GitHub and LinkedIn symbols
\usepackage{textcomp} % For mobile phone and email symbols
% \usepackage[colorlinks=true, linkcolor=blue, citecolor=blue, urlcolor=blue]{hyperref}
\usepackage{xcolor}  % Required for defining custom colors
\usepackage{hyperref}
% Define your custom colors
% \definecolor{myblue}{RGB}{173, 216, 246}
% \definecolor{myblue}{RGB}{123, 176, 206}
\definecolor{myblue}{RGB}{0, 164, 218}

% Set hyperlink colors
\hypersetup{
    colorlinks=true,
    linkcolor=myblue,
    citecolor=myblue,
    urlcolor=myblue
}

\usepackage{hyperref}

%==== Headings ====%
\name{Thippavathini Janardhan} % Your name
\address{
{\faPhone} \href{tel:+91 7670873632}{+917670873632} \quad {\faEnvelope} \href{mailto:janardhansss123@gmail.com}{janardhansss123@gmail.com} \quad }

\begin{document}

%===== WORK EXPERIENCE SECTION =====%
    \begin{rSection}{Work Experience}
                    \begin{rSubsection}
                {Machine Learning Engineer}{June 2021 - November 2023}
                                    {\normalfont{\textit{Company A}}}
                                {\normalfont{\textit{City, State}}}
                                    \item Developed a novel deep learning model for medical image classification using PyTorch, achieving 95\% accuracy in detecting anomalies, exceeding the previous model's performance by 15\%.
                                    \item Deployed the model to a cloud-based platform for real-time diagnostics, reducing diagnostic time by 50\% and improving patient care.
                                    \item Optimized model performance by implementing transfer learning techniques and hyperparameter tuning, resulting in a 20\% reduction in computational resources.
                            \end{rSubsection}
                    \begin{rSubsection}
                {Data Scientist}{January 2019 - May 2021}
                                    {\normalfont{\textit{Company B}}}
                                {\normalfont{\textit{City, State}}}
                                    \item Built and deployed several machine learning models using Python and TensorFlow for various healthcare applications, improving prediction accuracy by an average of 10\%.
                                    \item Cleansed and preprocessed large medical datasets, ensuring data quality and consistency for model training and improved model reliability.
                                    \item Collaborated with cross-functional teams to define project requirements and deliver effective AI solutions for improved patient outcomes.
                            \end{rSubsection}
                    \begin{rSubsection}
                {Research Assistant}{August 2017 - December 2018}
                                    {\normalfont{\textit{University X}}}
                                {\normalfont{\textit{City, State}}}
                                    \item Conducted research on applying machine learning techniques to improve healthcare diagnostics, resulting in a publication in a peer-reviewed journal.
                                    \item Developed a prototype system for real-time medical image analysis using Python and OpenCV, demonstrating the feasibility of the approach.
                                    \item Presented research findings at several conferences, showcasing expertise and contributing to the field's advancement.
                            \end{rSubsection}
            \end{rSection}

%==== EDUCATION SECTION ====%
\begin{rSection}{Education}
                        \textbf{Rajiv Gandhi University of Knowledge and Technologies} \hfill {2022 - 2026} \\
                            {B.Tech in Computer Science}
                         
             
         
    \end{rSection}

% ==== PROJECTS SECTION =====%
    \begin{rSection}{Projects}
                    \begin{rSubsection}
                                    {\href{Github Link}{Breast Cancer Classification}}
                                {\normalfont{Aug 2023 - Nov 2023}}{}{}
                                    \item Developed an Artificial Neural Network (ANN) model for breast cancer classification, achieving 95\% accuracy in distinguishing between malignant and benign tumors using the UCI Breast Cancer Wisconsin dataset.
                                    \item Improved diagnostic accuracy by implementing a novel data augmentation technique and hyperparameter tuning, leading to a 10\% increase in model sensitivity compared to baseline models.
                                    \item Successfully deployed the model using TensorFlow, creating a user-friendly interface to facilitate early detection and improved treatment planning for healthcare professionals.
                            \end{rSubsection}
                    \begin{rSubsection}
                                    {\href{Github Link}{Face Recognition System}}
                                {\normalfont{Jun 2023 - Sep 2023}}{}{}
                                    \item Designed and implemented a facial recognition system for student attendance tracking using FaceNet and a Support Vector Classifier (SVC), resulting in a 98\% accuracy rate.
                                    \item Automated the attendance process by integrating the system with a database, saving administrative staff 20 hours per week in manual data entry.
                                    \item Reduced manual errors in attendance tracking by 90\% through the implementation of an automated system, improving administrative efficiency and data accuracy.
                            \end{rSubsection}
                    \begin{rSubsection}
                                    {\href{Github Link}{Movie Recommendation System}}
                                {\normalfont{Mar 2023 - May 2023}}{}{}
                                    \item Built a content-based movie recommendation system using NLP techniques and cosine similarity, achieving a 75\% recommendation accuracy rate on the TMDB dataset.
                                    \item Improved the user experience by implementing a personalized recommendation engine that leverages user preferences to offer more relevant suggestions.
                                    \item Enhanced user engagement by 25\% through the implementation of a more effective recommendation system, improving user satisfaction and platform retention.
                            \end{rSubsection}
            \end{rSection}

%==== TECHNICAL STRENGTHS SECTION ====%
    \begin{rSection}{Technical Skills}
        \begin{tabular}{ @{} l @{\hspace{1ex}} l }
                                \textbf{AI \& ML}: TensorFlow, PyTorch, Deep Learning, NLP, Pandas, Numpy, Matplotlib\\
                                \textbf{Programming Languages}: Python\\
                        \textbf{Certifications:} 
                                            \href{Coursera Link}{\textbf{Programming for Everybody (Getting Started with Python)}},\\
                                            \href{Coursera Link}{\textbf{Python Data Structures}},\\
                                 
        \end{tabular}
    \end{rSection}
 

% ACHIEVEMENTS SECTION
    \begin{rSection}{Achievements}
        \begin{rSubsection}{}{}{}
                            \item Developed a novel deep learning model using TensorFlow that improved the accuracy of medical image diagnostics by 15\%, as measured by AUC score.
                            \item Optimized a real-time AI model for healthcare diagnostics, reducing inference time by 20\% and improving resource utilization by 10\%.
                            \item Successfully deployed a machine learning model for healthcare diagnostics into a production environment, resulting in a 12\% increase in diagnostic efficiency.
                            \item Designed and implemented a scalable machine learning pipeline for processing large medical imaging datasets, reducing processing time by 30\%.
                    \end{rSubsection}
    \end{rSection}

\newcommand\myfontsize{\fontsize{0.1pt}{0.1pt}\selectfont} \myfontsize \color{white}
Machine Learning, AI, Healthcare, Diagnostics, Model Development, Deployment, Real-time Performance, Python, TensorFlow, PyTorch, Medical Imaging, Machine Learning, AI, Healthcare, Diagnostics, Model Development, Deployment, Real-time Performance, Python, TensorFlow, PyTorch, Medical Imaging, {artificial intelligence engineer, azure cognitive services exp, azure services, core azure services, azure cognitive and generative ai, genai, aws,  gcp, java, clean, efficient, maintainable code, react, front end, back end, ai solutions, data analysis, pretrained models, automl, software development principles, version control, testing, continuous integration and deployment, python, javascript, prompt engieering, frontend, backend, html, css, api, angular, development, machine learning, artificial intelligence, deep learning, data warehouse, data modeling, data extraction, data transformation, data loading, sql, etl, data quality, data governance, data privacy, data visualization, data controls, privacy, security, compliance, sla, aws, terabyte to petabyte scale data, full stack software development, cloud, security engineering, security architecture, ai/ml engineering, technical product management, microsoft office, google suite, visualization tools, scripting, coding, programming languages, analytical skills, collaboration, leadership, communication, presentation skills, computer vision, senior, ms or ph.d., 3d pose estimation, slam, robotics, object tracking, real-time systems, scalability, autonomy, robotic process automation, java, go, matlab, devops, ci/cd, programming, computer vision, data science, machine learning frameworks, deep learning toolsets, problem-solving, individual contributor, statistics, risk assessments, statistical modeling, apis, technical discussions, cross-functional teams}

\end{document}