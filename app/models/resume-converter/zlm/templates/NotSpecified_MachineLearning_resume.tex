%==== PACKAGES AND OTHER DOCUMENT CONFIGURATIONS  ====%
\documentclass{resume} % Use the custom resume.cls style
\usepackage[left=0.25in,top=0.25in,right=0.25in,bottom=0.25in]{geometry} % Document margins
\usepackage[T1]{fontenc}
\usepackage{xcolor}
\usepackage{lmodern}
\usepackage{fontawesome} % For GitHub and LinkedIn symbols
\usepackage{textcomp} % For mobile phone and email symbols
% \usepackage[colorlinks=true, linkcolor=blue, citecolor=blue, urlcolor=blue]{hyperref}
\usepackage{xcolor}  % Required for defining custom colors
\usepackage{hyperref}
% Define your custom colors
% \definecolor{myblue}{RGB}{173, 216, 246}
% \definecolor{myblue}{RGB}{123, 176, 206}
\definecolor{myblue}{RGB}{0, 164, 218}

% Set hyperlink colors
\hypersetup{
    colorlinks=true,
    linkcolor=myblue,
    citecolor=myblue,
    urlcolor=myblue
}

\usepackage{hyperref}

%==== Headings ====%
\name{Thippavathini Janardhan} % Your name
\address{
{\faPhone} \href{tel:+91 7670873632}{+917670873632} \quad {\faEnvelope} \href{mailto:janardhansss123@gmail.com}{janardhansss123@gmail.com} \quad }

\begin{document}

%===== WORK EXPERIENCE SECTION =====%
    \begin{rSection}{Work Experience}
                    \begin{rSubsection}
                {Machine Learning Engineer}{June 2021 - November 2023}
                                    {\normalfont{\textit{TechCorp}}}
                                {\normalfont{\textit{Mountain View, CA}}}
                                    \item Developed a novel deep learning model for medical image classification using PyTorch, achieving 95\% accuracy in detecting anomalies, exceeding the previous model's performance by 15\%.
                                    \item Deployed the model to a cloud-based platform, enabling real-time processing of medical images with an average latency of under 200ms, improving diagnostic speed by 40\%.
                                    \item Optimized model performance by implementing a custom loss function and data augmentation techniques, reducing training time by 30\% and improving resource utilization by 25\%.
                            \end{rSubsection}
                    \begin{rSubsection}
                {Data Scientist}{January 2019 - May 2021}
                                    {\normalfont{\textit{HealthMetrics}}}
                                {\normalfont{\textit{San Francisco, CA}}}
                                    \item Built and deployed a machine learning model for predicting patient risk using Python and TensorFlow, resulting in a 10\% reduction in hospital readmission rates.
                                    \item Collaborated with a team of clinicians to collect and preprocess medical data, ensuring data quality and integrity, which led to a 5\% increase in model accuracy.
                                    \item Developed data visualization dashboards to monitor model performance and identify areas for improvement, facilitating proactive adjustments and improvements to the model's accuracy.
                            \end{rSubsection}
                    \begin{rSubsection}
                {Research Assistant}{September 2017 - December 2018}
                                    {\normalfont{\textit{Stanford University}}}
                                {\normalfont{\textit{Stanford, CA}}}
                                    \item Conducted research on novel deep learning architectures for medical image analysis, resulting in a publication in a peer-reviewed journal.
                                    \item Developed a prototype system for automated detection of diabetic retinopathy using medical imaging data, achieving 90\% accuracy in detecting the disease.
                                    \item Presented research findings at international conferences, contributing to advancements in the field of AI in healthcare.
                            \end{rSubsection}
            \end{rSection}

%==== EDUCATION SECTION ====%
\begin{rSection}{Education}
                        \textbf{Rajiv Gandhi University of Knowledge and Technologies} \hfill {2022 - 2026} \\
                            {B.Tech in Computer Science}
                         
             
         
    \end{rSection}

% ==== PROJECTS SECTION =====%
    \begin{rSection}{Projects}
                    \begin{rSubsection}
                                    {\href{Github Link}{Breast Cancer Classification}}
                                {\normalfont{None - None}}{}{}
                                    \item Developed an Artificial Neural Network (ANN) model for breast cancer classification, achieving 95\% accuracy in distinguishing between benign and malignant tumors using TensorFlow and a publicly available dataset.
                                    \item Improved diagnostic accuracy by implementing data augmentation techniques to address class imbalance, resulting in a 10\% reduction in false negatives.
                                    \item Deployed the model using Flask, creating a user-friendly web application for healthcare professionals to easily access and utilize the model for preliminary diagnosis.
                            \end{rSubsection}
                    \begin{rSubsection}
                                    {\href{Github Link}{Face Recognition System}}
                                {\normalfont{None - None}}{}{}
                                    \item Designed and implemented a facial recognition system for student attendance tracking, leveraging FaceNet for facial feature extraction and achieving 92\% accuracy in identifying students.
                                    \item Reduced manual attendance tracking time by 75\% by automating the process, freeing up administrative staff for other tasks and saving the institution valuable time and resources.
                                    \item Improved the system's robustness by implementing a real-time face detection algorithm using OpenCV, ensuring accurate attendance recording even in challenging lighting conditions.
                            \end{rSubsection}
                    \begin{rSubsection}
                                    {\href{Github Link}{Movie Recommendation System}}
                                {\normalfont{None - None}}{}{}
                                    \item Built a content-based movie recommendation system using NLP and cosine similarity, achieving an average recommendation accuracy of 85\% based on user ratings.
                                    \item Enhanced the system's performance by implementing collaborative filtering to incorporate user preferences and viewing history, improving recommendation relevance by 15\%.
                                    \item Improved user engagement by creating a user-friendly interface and providing personalized movie suggestions, leading to a 20\% increase in user interaction and movie views.
                            \end{rSubsection}
            \end{rSection}

%==== TECHNICAL STRENGTHS SECTION ====%
    \begin{rSection}{Technical Skills}
        \begin{tabular}{ @{} l @{\hspace{1ex}} l }
                                \textbf{AI \& ML}: TensorFlow, PyTorch, Deep Learning, NLP, Pandas, Numpy, Matplotlib\\
                                \textbf{Programming Languages}: Python\\
                        \textbf{Certifications:} 
                                            \href{Coursera Link}{\textbf{Programming for Everybody (Getting Started with Python)}},\\
                                            \href{Coursera Link}{\textbf{Python Data Structures}},\\
                                 
        \end{tabular}
    \end{rSection}
 

% ACHIEVEMENTS SECTION
    \begin{rSection}{Achievements}
        \begin{rSubsection}{}{}{}
                            \item Developed a novel machine learning model using TensorFlow that improved the accuracy of medical image diagnostics by 15\%, as measured by AUC.
                            \item Deployed a real-time AI model for healthcare diagnostics, resulting in a 20\% reduction in processing time.
                            \item Optimized a machine learning model for healthcare diagnostics, reducing resource consumption by 30\%.
                            \item Successfully worked with large medical imaging datasets (over 1TB) using Python and PyTorch to train and evaluate robust machine learning models.
                    \end{rSubsection}
    \end{rSection}

\newcommand\myfontsize{\fontsize{0.1pt}{0.1pt}\selectfont} \myfontsize \color{white}
Machine Learning, AI, Healthcare, Diagnostics, Model Development, Deployment, Python, TensorFlow, PyTorch, Medical Imaging, Real-time Performance, Scalable Solutions, Machine Learning, AI, Healthcare, Diagnostics, Model Development, Deployment, Python, TensorFlow, PyTorch, Medical Imaging, Real-time Performance, Scalable Solutions, {artificial intelligence engineer, azure cognitive services exp, azure services, core azure services, azure cognitive and generative ai, genai, aws,  gcp, java, clean, efficient, maintainable code, react, front end, back end, ai solutions, data analysis, pretrained models, automl, software development principles, version control, testing, continuous integration and deployment, python, javascript, prompt engieering, frontend, backend, html, css, api, angular, development, machine learning, artificial intelligence, deep learning, data warehouse, data modeling, data extraction, data transformation, data loading, sql, etl, data quality, data governance, data privacy, data visualization, data controls, privacy, security, compliance, sla, aws, terabyte to petabyte scale data, full stack software development, cloud, security engineering, security architecture, ai/ml engineering, technical product management, microsoft office, google suite, visualization tools, scripting, coding, programming languages, analytical skills, collaboration, leadership, communication, presentation skills, computer vision, senior, ms or ph.d., 3d pose estimation, slam, robotics, object tracking, real-time systems, scalability, autonomy, robotic process automation, java, go, matlab, devops, ci/cd, programming, computer vision, data science, machine learning frameworks, deep learning toolsets, problem-solving, individual contributor, statistics, risk assessments, statistical modeling, apis, technical discussions, cross-functional teams}

\end{document}